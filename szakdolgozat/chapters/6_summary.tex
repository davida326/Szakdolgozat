\Chapter{Összefoglalás}

A szakdolgozatom célja a Linux processz ütemező optimalizálása, megadott használati módokra.
A dolgozat elején bemutattam a jelenleg elterjedt processz ütemezőt. Úgy gondoltam, hogy fontos lehet megérteni, hogy hogyan müködik, és milyen lehetőségeink vannak a módosítására.

A módszer amit választottam a processz ütemező optimalizálására, az csak néhány lehetőséget fedett le a sok megközelítés közül. A rendszer optimalizálás témaköre mai napig is népszerű, sok könyv és kutatási eredmény jelenik meg ezzel kapcsolatban, amelyek különböző módszereket mutatnak be, a jobb teljesítmény eléréséhez a rendszereinkben.

Az elméleti rész után, az elkészült programokhoz felhasznált komponensek kerültek bemutatásra, majd a megvalósítási folyamat következett. A \textit{parameter-test} program számos módosításon átesett, először még csak előre meghatározott benchmark tesztekkel lehetett végezni méréseket és nem volt lehetőség konfigurációk kiválasztására sem. Azóta sikerült megoldanom a konfigurációs fájlok módosítását és a PTS program megfelelő paraméterezését. 

Az általam készített programok főként az ütemező hangoló változók módosításáról szóltak, mivel ezeket viszonylag egyszerű elérni, futás közben is módosíthatók, és a dinamikusan számolódó időszeletek miatt, úgy gondolom hogy egy megfelelő szabályzást érhetünk el vele a processz ütemezőn.

A programok jelenlegi állapotukban müködőképesek, azonban nyilván van lehetőség a fejlesztésére. A következő néhány pontba összegyűjtöttem az ehhez kapcsolódó lehetséges további célokat, irányokat.
\begin{itemize}
\item Machine Learning program tanítását összesen 6144 mintával végeztem, úgy hogy a változók invervallumaikat négy részre szedtem szét. Úgy gondolom, hogy több mintával és az invervallumra több részre való bontásával tovább fejleszthető a model, ami precízebb javaslatokhoz vezethez.
\item A \textit{parameter-test} programnál megoldható lenne a kiválasztott benchmark automatikus telepítése, mivel a jelenlegi változat csak elindítja a már feltelepített teszt programot.
\item Az ML programhoz célszerű lehet egy grafikus interfészt készíteni, remélve, hogy annak a segítségével több felhasználó számára  is egyszerűbbnek tűnhet a használata.
\end{itemize}

Összességében tehát az elkészült ML program alkalmas lehet olyan rendszergazda számára, aki esetleg régebbi számítógépekkel dolgozik, és ki szeretné belőlük hozni a maximum teljesítményt.
A ML program futtatásához csak néhány kérdés megválaszolására van szükség az adott felhasználási móddal kapcsolatban, és ez alapján javasol egy paraméterkészletet a processz ütemező hangolására.
