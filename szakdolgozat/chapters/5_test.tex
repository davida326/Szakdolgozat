\Chapter{Tesztelés}

A fejezet célja, hogy bemutassa és összehasonlítsa a teszteket, illetve az elért eredményeket. Mivel a benchmark programok használati módban eltérnek egymástól, megfigyelhető hogy egyes benchmark programoknál, jelentősebbek a kernel változók módosításainak hatásai.
Ezeket diagrammok és ábrák formájában összesítem a következőkben.

A benchmark tesztelési környeznek egy hagyományos asztali számítógépet használtam fel, amelynek hardveres és szoftveres specifikációi a következők.

\begin{lstlisting}
       _,met$$$$$gg.         davida@debian-pc 
    ,g$$$$$$$$$$$$$$$P.      ---------------- 
  ,g$$P"     """Y$$.".       OS: Debian GNU/Linux 10 x86_64 
 ,$$P'              `$$$.    Host: H81-D3 
',$$P       ,ggs.     `$$b:  Kernel: 5.8.5 
`d$$'     ,$P"'   .    $$$   Uptime: 2 hours, 25 mins 
 $$P      d$'     ,    $$P   Packages: 1669 (dpkg) 
 $$:      $$.   -    ,d$$'   Shell: bash 5.0.3 
 $$;      Y$b._   _,d$P'     Resolution: 1920x1080 
 Y$$.    `.`"Y$$$$P"'        WM: i3 
 `$$b      "-.__             Theme: Adwaita [GTK3] 
  `Y$$                       Icons: Adwaita [GTK3] 
   `Y$$.                     Terminal: urxvt 
     `$$b.                   CPU: Intel i5-4440 (4) 3.300GHz 
       `Y$$b.                GPU: AMD Radeon R7 265 1024SP 
          `"Y$b._            GPU: Intel HD Graphics 
              `"""           Memory: 1358MiB / 7844MiB
\end{lstlisting}
A neofetch program választottam erre a feladatra. Úgy gondoltam, hogy ez kellőképpen összegzi az szükséges adatokat. A Neofetch program, egy CLI-s rendszer, információkat megjelenítő eszköz, amelyet BASH-ben írtak [].

\Section{Benchmark tesztek elemzése}

Az \texttt{openbenchmarking.org} oldalon több mint 600 benchmark tesztprogram érhető el kategorizálva. A teszteléshez minden kategóriából választottam programokat, így szerettem volna szimulálni a különböző felhasználási módokat. Figyelembe kellett vegyem a benchmark választásnál azt is, hogy a programok viszonylag gyorsan végezzenek, mivel sok féle kombinációval kellett végrehajtanom ugyanazokat a teszteket. 

\SubSection{CPU intezív tesztek}

Az első felhasználási mód, a CPU-t terhelő benchmarkok kategóriája. Ehhez két tesztet is választottam, névszerint a \textit{sampleprogram}-ot illetve az \textit{ebizzy}-t.

Minden teszthez készítettem néhány diagrammot, amin megfigyelhető, hogy a különböző változók módosításával milyen eredmények születtek (\ref{fig:CpuParameters}. ábra).

\begin{figure}[h!]
\centering
\includegraphics[width=\textwidth]{images/cpuBenchmarkValues.png}
\caption{CPU benchmark eredményekhez tartozó kernel változók és prioritás értékek}
\label{fig:CpuParameters}
\end{figure}

Ahogy látható, a két benchmark teszteredményeit hasonlónak mondhatjuk. Mindkettőnél megfigyelhető egy "hézag", ami elválasztja a jobb és a rossz eredményeket. Az eredmények romlása akkor jelentkezik amikor a \texttt{sched\_wakeup\_granularity\_ns} változóhoz nagyobb értékét állítunk, mint a \texttt{sched\_latency\_ns} fele. Ekkor sajnos a kis időciklusú taszkok nem lesznek képesek versenyezni a többi folyamattal amikor a processzort éppen nagy terhelés alatt áll.
Az eredmények nagy része a legjobb kategóriába került, ez a \textit{sampleprogram}-nál 53\%, amíg az \textit{ebizzy}-nél pedig 58\%. Ebből következtethetünk arra, hogy a CPU intenzív felhasználási módhoz a legtöbb beállítás a kernel változókon, elfogadható teljesítményt eredményezett.

\SubSection{Rendszert terhelő benchmark}

Ebben a kategóriában nem processzor kimerítés hanemm, inkább a nagy rendszerhívás szám jellemzőbb.
A született eredmények nagyrésze a legjobb kategóriába került. Így azt mondhatom hogy, a intenzív rendszerterhelés felhasználási módhoz a legtöbb beállítás amit végeztem a kernel változókon, megfelelő teljesítményt eredményez. Minél több eredmény kerül a legjobb kategóriába, az ML program annál könnyebben fog tudni "jobb" értéket javasolni az adott felhasználási módhoz.

\SubSection{Videókártyát és memóriát terhelő benchmarkok}

A grafikus és memória benchmarkok, felhasználási módban és terhelésben is eltérnek egymástól, viszont mégis van bennük egy közös.
A grafikus benchmarkhoz készített \ref{fig:GraphicsAndMemoryParameters}. ábrán (ami a változó értékeket jeleníti meg és az azokkal elért eredményeket) látható, hogy itt is jelen van a CPU tesztnél már megfigyelt "hézag", a jobb illetve rosszabb értékek között, ami viszont a memória benchmarknál nem jelentkezik.

\begin{figure}[h!]
\centering
\includegraphics[scale=0.35]{images/graphicsAndMemoryBenchmarkValue.png}
\caption{Grafikus és memória benchmark eredményekhez tartozó kernel változók és prioritás értékek}
\label{fig:GraphicsAndMemoryParameters}
\end{figure}

A memória és grafikus benchmark közös tulajdonsága az eredmény értékek eloszlása. Mindkét benchmarknál elmondható, hogy az eredmények legnagyobb része a legrosszabb kategóriába került.
Ezek az értékek \aref{fig:GraphicsAndMemoryChart}.  kördiagrammokról leolvashatók.

\begin{figure}[h!]
\centering
\includegraphics[width=\textwidth]{images/graphicsAndMemoryBenchmarkChart.png}
\caption{Grafikus és memória benchmarkokból származó értékek}
\label{fig:GraphicsAndMemoryChart}
\end{figure}

Így a 3-as, 4-es kategóriába tartozó eredmények ritkának mondhatók. Amennyiben megfigyeljük az adathalmazt, hogy mi is váltotta ki ezeket, észrevehetjük hogy a memória benchmarknál minden "jobb" eredmény eléréséhez szükség volt a szerver szintű terhelési mód beállítására.
A grafikus benchmarknál sajnos nem találtam szabad szemmel látható összefüggéseket, így a legjobb beállítási értékek megválasztását az ML programra bíztam.

\SubSection{Háttértárat terhelő benchmark}

A háttértárakat terhelő benchmark kategória tipikusan az, ahol a processzek az életük nagyrészét I/O-ra történő várakozzással töltik.
Az itt kapott eredmények nagy része 1-es illetve 2-es kategóriába került, ezeket vehetjük úgymond átlagos értékeknek (\ref{fig:diskChart}. ábra). Mivel a legtöbb érték a legrosszabb eredmények halmazaiba kerültek, feltételezhetjük hogy néhány beállítás elősegítette a jobb értékek kialakulását. 
Fontos kiemelnem hogy, a módosított változó értékek nem függetlenek egymástól, emiatt nem feltételezhetjük azt, hogy ha a legjobb értékekeket elért változókat kiválasztjuk, megkapjuk a tökéletes beállítást az ütemező hangolására.

\begin{figure}[h!]
\centering
\includegraphics[scale=0.8]{images/diskBenchmarkValue.png}
\caption{Háttértárakat terhelő benchmarkból származó értékek.}
\label{fig:diskChart}
\end{figure}

\Section{Tuned profilok összehasonlítása}

A korábban már említett Tuned rendszer hangoló szolgáltatás segítségével, hangolhatunk a rendszerbeállításainkon, kiválasztott profilok szerint.
A tuned jelenlegi változtában viszonylag sok profil elérhető, különböző felhasználási módokhoz, ezért kiválasztottam néhányat amelyek a következők.
\begin{itemize}
\item desktop

Ez a profil egyszerű asztali felhasználási módra próbálja optimalizálni a rendszert.
Mindemellett engedélyez egy \texttt{scheduler autogroups} opciót, amivel jobb válaszidőt képes elérni, ez főként interaktív folyamatok számára lényeges.

\item latency-performance

Ez egy tipikus szerver profil, ami alacsony késleltetésre van optimalizálva. Kikapcsol mindenféle energiamegtakarítási mehanizmust és a sysctl segítségével javítja a késleltetést.

\item throughput-performance

Ez szintén egy szerver profil, ami a magas áteresztőképességre van optimalizálva.
Szintén kikapcsolja az energiamegtakarító mehanizmusokat és a sysctl felhasználásával javítja a hálózati I/O és háttértárak teljesítményét, emellett próbálja a deadline ütemezési osztályt is használni.
\end{itemize}

Ezeket a profilokat hasonlítom össze egymással és egy olyan változattal ahol nem használok tuned optimalizálást (Teljesen általános felhasználási mód).
Az utóbbit \texttt{none} elnevezéssel fogom jelölni a diagrammokon.
A teszteléshez a már korábban bemutatott benchmarkok közül fogom felhasználni az sample-programot, ebizzy-t, fs-mark-ot és a ctx-clock-ot.
A következő ábrákon láthatjuk hogy a desktop profil teljesített a legjobban, a legtöbb teszten.
\begin{figure}[h!]
\centering
\includegraphics[width=\textwidth]{images/sampleProgramAndCtxClock.png}
\caption{Sample-program és Ctx-Clock benchmark eredmények az említett profilokkal}
\label{fig:tunedProfilesSampleprogramAndCtxClock}
\end{figure}

Itt a kisebb értékek számítanak jobbnak és a sárgával jelöltem a legjobb értéket elért profilt.

\begin{figure}[h!]
\centering
\includegraphics[width=\textwidth]{images/ebizzyAndFsMark.png}
\caption{Ebizzy és FS-Mark benchmark eredmények az említett profilokkal}
\label{fig:tunedProfilesEbizzyAndFsmark}
\end{figure}

Az ebizzy és fs-mark esetén pedig a nagyobb értékek jelölik a jobb eredményt.
Egy adott profillal és egy benchmark programmal tíz mintát készítettem és ezeknek vettem az átlagukat.

\subsection{Tuned profilok összevetése a \textit{SchedulerTuneML} program felhasználási módjaival}

A \textit{SchedulerTuneML} programban profilok helyett felhasználási módokat lehet szabályozni, ami alapján tudjuk hangolni a rendszerünket. A tuned programban már bemutatott \texttt{desktop} és \texttt{latency-performance} profilokhoz hasonló felhasználási módokat készítettem a \textit{SchedulerTuneML} programmal. A \texttt{desktop} profilhoz hasonlító felhasználási mód kialakításához, több processzormagos felhasználást, nem szerver szintű terhelés, magas prioritást és aktív swap használatot alkalmaztam. A \texttt{latency-performance} profil megközelítéséhez, szerver szintű terhelést, több processzormagos felhasználást, alacsony prioritást választottam, swap használat nélkül.

A különböző profilok és felhasználási módok teljesítményeinek összevetésére, szintén a sample-program, ebizzy, fs-mark és a ctx-clock benchmarkot fogom felhasználni. 

\begin{figure}[h!]
\centering
\includegraphics[width=\textwidth]{images/tunedAndSchedulerTuneMLCompareSampleprogramCtxclock.png}
\caption{Tuned profilok és \textit{SchedulerTuneML} felhasználási módok összevetése (Sample-program és Ctx-clock benchmark)}
\label{fig:tunedProfilesComparedToSchedulerTuneMlSampleprogramAndCtxClock}
\end{figure}

Sárgával jelöltem a legjobb értéket elérő hangolási módokat. Az első benchmarknál megfigyelhető, hogy a \textit{SchedulerTuneML} program Szerverhez hasonlító felhasználási módja, nagyon rossz értéket produkált.

\begin{figure}[h!]
\centering
\includegraphics[width=\textwidth]{images/tunedAndSchedulerTuneMLCompareEbizzyAndFsMark.png}
\caption{Tuned profilok és \textit{SchedulerTuneML} felhasználási módok összevetése (Ebizzy és FS-Mark benchmark)}
\label{fig:tunedProfilesComparedToSchedulerTuneMlEbizzyAndFsMark}
\end{figure}

Az első benchmarkot követő teszteken, a \textit{SchedulerTuneML} Szerver felhasználási módja érte el a legjobb eredményeket.